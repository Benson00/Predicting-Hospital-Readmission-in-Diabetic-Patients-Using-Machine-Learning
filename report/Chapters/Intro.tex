\section{Introduction}

\subsection{Context and Motivation}
A large proportion of individuals diagnosed with diabetes mellitus, a leading chronic non-communicable disease, experience recurrent hospital admissions due to insufficient control of their condition. The term "readmission" denotes the return of a patient to the same hospital department within a defined period for issues related to the same underlying disease. Such readmissions are often unplanned and may occur from various factors, such as misdiagnosis during the initial visit, disease recurrence, early discharge, or other clinical complications \cite{dungan2012effect, eby2014predictors}. These events not only affect patient health outcomes but also contribute significantly to the rising costs of healthcare systems. To mitigate these issues, it is crucial to identify patients at high risk of readmission before discharge, enabling timely interventions and improved care planning. Achieving this requires high-quality clinical data, which forms the basis for predictive modeling.


\subsection{Data provenance}
The provenance of the data used in this study derives from the \textit{Diabetes 130-US hospitals for years 1999–2008} dataset, which is publicly available on the UCI Machine Learning Repository \cite{uciml}. Each record corresponds to a single inpatient hospital encounter that meets the following criteria: 
\begin{enumerate}
    \item the encounter must be an inpatient admission
    \item the diagnosis must include any form of diabetes
    \item the length of stay must be between 1 and 14 days
    \item at least one laboratory test must have been performed
    \item at least one medication must have been administered during the encounter
\end{enumerate}

\subsection{State of the art}
Hospital readmission among patients with diabetes mellitus is a well-recognised challenge in clinical practice, carrying substantial implications for both patient outcomes and healthcare costs. Consequently, this topic has been the subject of extensive research, with efforts directed towards identifying key risk factors and developing predictive models to anticipate and ultimately reduce preventable readmissions.\\
Using the same dataset employed in the present work, a previous study investigated the prediction of 30-day hospital readmission among diabetic patients using machine learning (ML) techniques. The authors applied several well-established algorithms, including logistic regression, decision trees, random forests, achieving a maximum area under the ROC curve (AUC) of approximately 0.68 \cite{shang202130}.\\
Deep neural networks (DNNs) were also explored in this context, yielding a substantially higher AUC of 0.90 \cite{hammoudeh2018predicting}. However, the black-box nature of these models and the resulting lack of interpretability may limit their clinical adoption.\\
In this study, we revisit the problem using traditional ML models, with the addition of data engineering and hyperparameter optimisation, to evaluate whether such approaches can match or surpass the performance of existing models while maintaining interpretability.


\subsection{Objective}
The primary objective of this study is to develop and compare machine learning (ML) methods for predicting the risk of hospital readmission in diabetic patients using the previously described dataset. Unlike prior research that has primarily focused on predicting 30-day readmissions, this work aims to classify whether a patient will be readmitted at any point in the future, without imposing a temporal constraint.\\
The proposed approach applies traditional ML models commonly employed in earlier studies, enhanced through the integration of hyperparameter optimization and feature selection techniques.



