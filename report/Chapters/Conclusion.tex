\section{Conclusion}

In this study, we evaluated multiple machine learning models on a clinical dataset, comparing performance with the full feature set and with a reduced, feature-selected subset. Across all experiments, the Random Forest classifier consistently achieved the strongest results. Feature selection provided only marginal improvements, suggesting that the original features already capture most of the predictive signal and that the dataset itself may have intrinsic limitations in predicting the target outcomes.\\
ROC and AUC analyses reinforced these findings: Random Forest achieved the highest discriminative performance (AUC $\approx$ 0.69), closely matching state-of-the-art reports of a maximum AUC of ~0.68. Notably, all tested models, including state-of-the-art approaches, yielded comparable performance, underscoring the dataset’s inherent constraints.\\
Future work should therefore prioritize collecting richer and more informative data, as the current feature set appears insufficient to drive substantial improvements in predictive accuracy. While the present results are reasonable within these constraints, higher-quality data will likely be necessary to enhance model performance and discriminative power.\\
Finally, analysis of feature importance highlights the clinical implications of ensuring close patient monitoring and robust post-discharge follow-up as potential strategies to reduce readmission rates.
