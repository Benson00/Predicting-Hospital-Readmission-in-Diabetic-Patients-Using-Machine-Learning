\section{Conclusion}

In this study, we evaluated several machine learning models on a clinical dataset using both the full feature set and a feature-selected subset. Across all experiments, the Random Forest classifier consistently demonstrated the best performance, confirming its robustness and ability to generalize.\\\\
Feature selection led to only marginal improvements, indicating that most predictive information is already captured in the original features and that the dataset may have intrinsic limitations in accurately predicting the target outcomes. ROC and AUC analyses further confirmed these observations: Random Forest achieved the highest discriminative performance (AUC $\approx$ 0.69), which is in line with state-of-the-art results reporting a maximum AUC of approximately 0.68.\\\\
All tested models, including those from the state-of-the-art, achieved similar performance, pointing to intrinsic limitations in the dataset. Future work should focus on collecting a more informative dataset, as the current features may not be sufficient to substantially improve predictive performance. While the current results are reasonable, better-quality data may be required to achieve higher accuracy and discriminative power.\\\\
Additionally, analyzing the feature importances emphasizes the need for careful monitoring of patients and ensuring adequate post-discharge follow-up to reduce readmission rates. 
