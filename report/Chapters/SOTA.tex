\section{Related Work}

Hospital readmission among patients with diabetes mellitus is a well-recognised challenge in clinical practice, carrying substantial implications for both patient outcomes and healthcare costs. Consequently, this topic has been the subject of extensive research, with efforts directed towards identifying key risk factors and developing predictive models to anticipate and ultimately reduce preventable readmissions.\\\\
Using the same dataset employed in the present work, a previous study investigated the prediction of 30-day hospital readmission among diabetic patients using machine learning (ML) techniques. The authors applied several well-established algorithms, including logistic regression, decision trees, and support vector machines, achieving a maximum area under the ROC curve (AUC) of approximately 0.68 \cite{shang202130}.\\\\
Deep neural networks (DNNs) were also explored in this context, yielding a substantially higher AUC of 0.90 \cite{hammoudeh2018predicting}. However, the black-box nature of these models, and the resulting lack of interpretability, may limit their clinical adoption.\\\\
In this study, we revisit the problem using traditional ML models, combined with extensive data engineering and hyperparameter optimisation, to evaluate whether such approaches can match or surpass the performance of deep learning models while maintaining interpretability.
